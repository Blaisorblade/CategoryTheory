\documentclass{article}
\usepackage{hyperref}

\newcommand{\from}{\leftarrow}
\newcommand{\ap}{\;}
\newcommand{\Gl}{\lambda}

\begin{document}
\section{Intro}
My personal notes on category theory.
Right now, for my own personal consumption only.
\section{Notation}
The argument order for function composition is awkward in category theory. It
makes sense to write $f \circ g = f(g(x))$, but that means ``do $g$, then $f$``.
In diagrams, you usually write that by having $g$ to the left and $f$ to the right.

To obviate for this, Stephen Diehl writes his diagrams the other way around
(without warning)%
\footnote{\url{http://www.stephendiehl.com/posts/adjunctions.html}}.
%
This way, the inputs are to the right and outputs are to the left, just as for
function composition and application to the arguments. Similarly, in vertical
composition of things (natural transformations, which we'll get to much later),
inputs are to the bottom and outputs to the top; moving the bottom to the right
(a counterclockwise rotation) shows where the two things to compose go.

To match diagrams, we'll indicate domain and codomain with the notation $f: A
\from B$, instead of $f: B \to A$. Moreover, we'll still assign names from left
to right, so $f: A \from B$ instead of $f: B \from A$. This also matches with
application: if $f: A \from B$, when we look at $f\ap b$, we see that the thing to
the right, $b$, has type $B$, while the thing ``to the left'', $f\ap b$, has type
$A$.

However, existing notations have the standard meaning, and even $A \from B$ has
the ``obvious'' meaning one would expect, that is, $B \to A$.

\paragraph{Function composition}
We can define the function composition $f \circ g: A \from C$ of $f: A \from B$
and $g: B \from C$ as the function $\Gl x. f\ap (g\ap x)$. Since the order of
function composition is intended to match application, and since we matched the
order of application with the types, we also automatically match the order of
composition with the types. In other words, the type of $\circ$ is now
\[(f : A
\from B) \circ (g: B \from C) : A \from C,\]
instead of
\[(f : B \to A) \circ (g: C \to B) : C \to A.\]
%
Alternatively, we can write
\[(\circ): (A \from C) \from (A \from B, B \from C).\]

What's notationally cool is that the two $B$ appear next to each other.

Unfortunately, the notation is not so helpful for currying, because we'd get the
first argument to the right:
\[(\circ) : (A \from C) \from (B \from C) \from (A \from B)\]


\paragraph{Functor composition}
The usual notation for composition works nicely, as long as we write functors $F
: C \from D$.

\paragraph{Composition of natural transformations}

% No, the usual notation is the right one for these diagrams. Diagrams with the
% usual arrows (left to right) would match with inverted function application,
% which would be too crazy.
%
%We write function application
%${}_xf$
\end{document}
%%% Local Variables:
%%% mode: latex
%%% TeX-master: t
%%% End:
